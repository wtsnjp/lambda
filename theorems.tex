%!uplatex
\documentclass[uplatex,dvipdfmx,report,fleqn]{jsbook}

%%
%% This is file `settings.tex' of lambda.
%%

% Preset
\usepackage[mathnote]{wtpreset}

% Set document infomation
\setdocinfo{
  title = 道楽的ラムダ計算,
  author = ワトソン}

% Settings
\usepackage{bussproofs}
\usemathlibrary{set, analysis, lambda}

% Commands
\newcommand{\tox}{\rightarrow_x}
\newcommand{\toxs}{\twoheadrightarrow_x}
\newcommand{\beq}{=_\beta}

% All done
\endinput

%% EOF
\DeclareSub[2]

\begin{document}

\chapter{ラムダ計算の性質}

\begin{abstract}

\end{abstract}

\section{チャーチ・ロッサーの定理}

\subsection{チャーチ・ロッサーの定理}

ラムダ項$M$の中に複数のベータ基が含まれている場合,そのいずれに注目するかによって何通りかの
ベータ簡約を行える可能性がある.例えば,$\fab{x}{xx}\fap{\fab{y}{y}z}$は
%
\begin{align}
\fab{x}{xx}\underline{\fap{\fab{y}{y}z}} &\bred \fab{x}{xx}z = M_1 \eqlabel{M1} \\
\underline{\fab{x}{xx}\fap{\fab{y}{y}z}} &\bred \fap{\fab{y}{y}z}\fap{\fab{y}{y}z} = M_2
\eqlabel{M2}
\end{align}
%
の2通りに簡約できる.上記2式では,左辺の注目したベータ基に下線を引いている.ここでベータ簡約
の結果にまたいくつかのベータ基がある場合,そのうちのいずれか1つを選んで簡約するということを
次々に続けていくと,1つのラムダ項から幾通りもの変換列ができる場合がある.
%その際
%\[
%\fab{x}{a}\Omega \bred \fab{x}{a}\Omega \bred \fab{x}{a}\Omega \bred \cdots
%\]
%のようにベータ簡約が新しいベータ基を生み,それをベータ簡約すると再び新しいベータ基が
%生まれるという状況がいつまでも続く場合が考えられ,ベータ変換列は必ずしも有限で終わる
%とは限らない.

そこで,このように1つのラムダ項から始まる様々な変換列がいずれ1つのラムダ項に
収束するのか,あるいは互いに無関係なラムダ項へと発散していくのか,つまり計算の順序は
計算結果に違いを与えるのか否かという疑問が生じる.本節で導くことになるチャーチ・ロッサーの
定理は,この問題に対して「計算の順序は計算結果に影響を与えない」という肯定的解答を与える
ものである.

まず,一般のラムダ項に関する二項関係$\tox$について合流性という概念を導入する.なお,$\tox$を
有限回繰り返して到達できることを$\toxs$と書く.
%
\begin{definition}[合流性]
任意のラムダ項$M,P,Q$について
\[
M \tox P \land M \tox Q \then \exists N.(P \tox N \land Q \tox N)
\]
が成り立つとき,二項関係$\tox$は\emph{1ステップ合流性}をもつ
という.また
\[
M \toxs P \land M \toxs Q \then \exists N.(P \toxs N \land Q \toxs N)
\]
が成り立つとき,二項関係$\tox$は\emph{複数ステップ合流性}をもつという.
\end{definition}
%
\begin{lemma}\thlabel{合流性}
二項関係$\tox$が1ステップ合流性をもつならば,$\tox$は複数ステップ合流性をもつ.
\end{lemma}
%
\begin{proof}
二項関係$\tox$が1ステップ合流性をもつとき,その合流状況を$M\toxs P, M\toxs Q$の変換列をもとに
\figref{合流状況}のように格子状に表現することができる.したがって$P \toxs N, Q \toxs N$
を満たす$N$が存在し,$\tox$は複数ステップ合流性をもつ.\qed
\end{proof}
%
\begin{figure}[b]
\centering
\imgsource{confluence}
\caption{1ステップ合流性が成り立つ場合の合流状況}
\figlabel{合流状況}
\end{figure}

補題\thref{合流性}の逆は成り立たない.実際,本節冒頭で示した式\eqref{M1, M2}について
$M_i\bred N\;(i=1,2)$なるラムダ項$N$は存在しない.この事実はベータ簡約の複数ステップ
合流性を主張するチャーチ・ロッサーの定理の証明を難しくする.そのためチャーチ・ロッサーの定理
を示すにはいくつか新しいアイデアが必要となる.
%
\begin{enumerate}
\item ラムダ項間に新たな二項関係$\pred$を導入する.
\item $\pred$と$\bred$は複数ステップでは同等になることを示す.
%\[
%M \preds N \iff M \breds N
%\]
\item $\pred$の1ステップ合流性を示す.
\end{enumerate}
%
ここで,二項関係$\pred$はあるラムダ項中に存在するすべてのベータ基を1回の操作で同時に
簡約してしまうというベータ簡約の拡張概念であり,形式的には次のように定義される.
%
\begin{definition}[並行簡約]
以下を満たすラムダ項間の二項関係$\pred$を\emph{並行簡約}という.
%
\begin{enumerate}
\item $a$が変数のとき$a\pred a$である.
\item $A\pred B$かつ$C\pred D$ならば$AC\pred BD$である.
\item $A\pred B$かつ$C\pred D$ならば$\fab{a}{A}C\pred B[D/a]$である.
\item $A\pred B$ならば$\fab*{a}{A}\pred$である.
%\item 以上に該当しなければ$M\pred N$ではない.
\end{enumerate}
\end{definition}
%
ここで上記定義内の規則は,\figref{並行簡約}に示す4つの推論規則として記述することもできる.
%
\begin{figure}
\centering
\begin{minipage}{.35\textwidth}
\centering
\imgsource{parallel1}
\end{minipage}
\begin{minipage}{.35\textwidth}
\centering
\imgsource{parallel2}
\end{minipage}
%
\\
%
\begin{minipage}{.35\textwidth}
\centering
\imgsource{parallel3}
\end{minipage}
\begin{minipage}{.35\textwidth}
\centering
\imgsource{parallel4}
\end{minipage}
\caption{並行簡約の規則}
\figlabel{並行簡約}
\end{figure}

\section{正規化定理}

\NeedsRevision

\end{document}
